%
% File amta08.tex
%
% Contact: nasmith@cs.cmu.edu

\documentclass[11pt]{article}
\usepackage{acl08}
\usepackage{times}
\usepackage{latexsym}
\setlength\titlebox{6.5cm}    % Expanding the titlebox

\title{L3, now with classifiers for WSD}

\author{
  Anonymous A \\
  School of Semiotic Countermeasures\\
  Foo University \\
  Anytowne \\
  {\tt anona@sc.foou.edu}\And
  Anonymous B \\
  School of Semiotic Countermeasures\\
  Foo University \\
  Anytowne \\
  {\tt anonb@sc.foou.edu}}

\date{}

\begin{document}
\maketitle
\begin{abstract}
  This document contains the instructions for preparing a camera-ready
  manuscript for the proceedings of AMTA-2010. The document itself conforms to
  its own specifications, and is therefore an example of what your manuscript
  should look like.  Authors are asked to conform to all the directions
  reported in this document.
\end{abstract}


\section{Introduction}
In this paper we describe extensions to the L3 translation system for
under-resourced languages.

L3 seeks to address the problem of the Linguistic Digital Divide.

* why is ambiguity a problem for RBMT?


* background with XDG


* how to integrate classifiers with L3


* add possible output words as features (does that help? we think it shouldn’t,
and can demonstrate with this experiment that ...)


SMT systems have used classifiers for CL-WSD and CL-PSD with varying success;
it seems to help somewhat. The language model for the target language is
helpful in local disambiguations, largely due to the left-to-right decoding
process.

However, in a rule-based system, and particularly in the case of translation
for under-resourced languages...

\section{Cross-Language Word Sense Disambiguation}
CL-WSD is distinct from the well-studied monolingual case, in that in the
cross-language case, the senses of a word in the target language are taken to
be only those that are distinct word types in the target language. Thus, for
example, if a word in the source language is always translated as the same word
in the target language, then there is no ambiguity to resolve, even if 

This approach has a long history in the translation literature, dating back to
Brown et al 1991...

More recently, Carpuat and Wu have integrated this into modern phrase-based SMT
systems ...  \cite{carpuatpsd} \cite{improvingsmtwsd}

\section{Classifier development}
For these experiments, we generated XXX sentences of English/Amharic bitext
from a simple grammar with lexical preferences baked in, such as those shown in
Figure XX, then trained 

\section{Classifier integration}

\section{Results}
The classifiers were able to learn the appropriate lexical preferences with
XX\% accuracy. They differed from the most frequent sense in XX\% of the time,
showing that they in fact were able to learn something useful.

On a modern desktop computer, the system took XXX seconds to return the first
translation. As judged by one of the authors, the most preferable translation
was produced as the top-scored translation, and first, in XX of the cases.

\section{Ongoing Work}
Clearly in the future our classifiers should be trained on real-world data
instead of a generated toy corpus. We would like to continue generalizing this
work to apply to other languages instead of just English/Amharic ...


In the medium term, we will integrate our RBMT system with interfaces for
computer-aided translation (CAT) in order to produce documents in
under-resourced languages 

\section{Conclusions}
Here we have presented  ...

more things to reference:

Gasser 2012: Toward a Rule-Based System for English-Amharic Translation
Rudnick 2011: Towards CL-WSD for Quechua

\section*{Acknowledgments}

Do not number the acknowledgment section.

\bibliographystyle{acl}
\bibliography{integrate.bib}{}

\end{document}
